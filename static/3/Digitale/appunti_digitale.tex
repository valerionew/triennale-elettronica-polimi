\documentclass{article}
\usepackage[utf8]{inputenc}
\usepackage[italian]{babel}
\usepackage{verbatim}
\usepackage{geometry}
\usepackage{enumitem}

\geometry{left = 2 cm, right = 2 cm, bottom = 3 cm, top = 3 cm}
\setlength{\parindent}{0em}

\title{Roba di Elettronica Digitale}
\author{Lorenzo Rossi}
\date{AA 2019/2020}

\begin{document}
\maketitle

\section{Dei tempi di propagazione e contaminazione}
\begin{itemize}
	\item Il tempo di contaminazione non potrà \textbf{mai} essere \textbf{superiore} al tempo di propagazione
	\item Restrizioni periodo di clock \[T_{clock} \geq \sum T_{pd}+ \sum T_{setup} - T_{skew} \]
	\item Restrizioni tempo di contaminazione \[\sum T_{cd} \geq \sum T_{hold} + T_{skew} \]
	dal si deduce che, per un singolo flip-flop, si traduce in  \[T_{hold} \leq T_{cd} - T_{skew} \]
	\item Frequenza massima di funzionamento \[f_{max} \leq \frac{1}{T_{ck_{max}}}\]
	\item Se un registro ha \textit{skew} pari a \(\Delta T\), il suo ritardo di propagazione \(T_{pd}\) aumenterà di un valore pari a \(\Delta T\)
	\item In un registro è possibile definire
	\begin{enumerate}
		\item \textit{Periodo di setup} \(T_{setup}\) che garantisce che il dato sia propagato attraverso il percorso di feedback prima che il master si chiuda \( \rightarrow \) implica che il dato sia memorizzato
		\item \textit{Periodo di hold} \(T_{hold}\) che garantisce che il dato sia stabile prima di permettere che il dato commuti
	\end{enumerate}
	\item A volte, i periodi di \textit{contaminazione} e \textit{propagazione} di un blocco logico/combinatorio sono indicati come \[T_{cd} \leq T_{ck}\rightarrow Q \leq T_{pd} \]
	\end{itemize}

\section{Dei circuiti lenient}
\begin{itemize}
	\item Un circuito \textit{non lenient} darà luogo a \textit{glitch} nell'uscita, cioè commutazioni non volute (uscite non valide in corrispondenza di ingressi validi)
	\item Nella \textit{mappa di Karnaugh,} un circuito \textit{non lenient} mostra due \textit{mintermini non essenziali} affacciati tra di loro 
	\item Se la \textit{f} mostra dei mintermini che \textbf{differiscono di un solo letterale negato tra di loro} allora il circuito che la implementa \textbf{non sarà lenient} 
	\item Per garantire che un circuito sia \textit{lenient} bisogna garantire che ci sia una linea che realizza \(f(y)=1\)
\end{itemize}

\section{Delle funzioni logiche realizzate con Multiplexer}
Per risolvere questa categoria di esercizi bisogna guardare la tabella della verità ponendo particolare attenzione agli ingressi del tipo \(100, 101\) che differiscono di un solo letterale e danno origine ad una uscita alta 

\section{Delle ROM}
\begin{itemize}
	\item In tutti gli esercizi trattati in aula, realizziamo le rom tramite \textit{nMos}, quindi di fatto andiamo a realizzare le reti di \textit{pull down} che realizzano \(f (a, b, c, ... ) = \bar{y} \)
	\item Cerchiamo sempre, quindi, quale combinazione degli ingressi \textbf{abbassa, portando a massa} la linea per poi applicare \textit{De Morgan} o l'analogo \textit{bubble pushing}
	\item Affinché una rete di \textit{pull down} funzioni, tutte le righe devono essere basse (quindi le funzioni trovate devono essere \textit{moltiplicate} tra di loro
\end{itemize}

\section{Delle FSM}
Deduzione del funzionamento della macchina dal diagramma degli stati: \textit{senza un numero massimo di stati non posso essere sicuro di aver esplorato ogni comportamento}. Solo conoscendo il numero di stati posso dedurre con correttezza il suo funzionamento.
Due \textit{FMS} si dicono \textbf{equivalenti} se e solo se ad ogni sequenza di ingressi corrispondono uguali sequenze di uscita
\subsection{Macchina di Moore}
\begin{itemize}
	\item L'uscita dipende solo dallo stato attuale \(output=f(stato\:attuale)\)
	\item Se l'ingresso cambia, l'uscita non cambia
	\item Reagisce agli ingressi più lentamente
	\item Nuovo stato ed uscita sincroni
	\item Rispetto ad una macchina di \textit{Mealy}, ha:
	\begin{enumerate}
		\item Maggior numero di stati
		\item Maggiori richieste hardware
		\item Design \textit{(progettazione)} facile
	\end{enumerate}
\end{itemize}

\subsection{Macchina di Mealy}
\begin{itemize}
	\item L'uscita dipende dallo stato attuale e dall'ingresso attuale \(output=f(stato\:attuale, ingresso\:attuale)\)
	\item Se l'ingresso cambia, l'uscita cambia
	\item Reagisce agli ingressi in modo rapido
	\item Nuovo stato ed uscita asincroni
	\item Rispetto ad una macchina di \textit{Moore}, ha:
	\begin{enumerate}
		\item Minor numero di stati
		\item Minori richieste hardware
		\item Design \textit{(progettazione)} più difficile
	\end{enumerate}
\end{itemize}

\section{Delle FSM in VHDL}
\begin{itemize}
	\item Ogni diagramma degli stati dovrà prevedere uno \textit{stato iniziale} nel quale la macchina finirà quando viene rilevato il segnale di reset
	\item Sarà necessario creare un \textit{tipo enumerato} che contenga tutti i possibili stati
	\item Sarà necessario creare 3 processi
	\begin{enumerate}
		\item Un \textit{process sincrono} che abbia \textit{clock} e \textit{reset} nella \textit{sensitivity list} e che si preoccupi di resettare la la macchina in corrispondenza del \textit{reset} e di aggiornare lo stato prossimo sul fronte positivo del \textit{clock}
		\item Un \textit{process asincrono} che abbia il \textit{segnale di ingresso} nella \textit{sensitivity list} e che si preoccupi di aggiornare lo stato prossimo in funzione dello stato attuale e dell'ingresso
		\item Un \textit{process asincrono} che abbia lo \textit{stato corrente} nella \textit{sensitivity list} e che si preoccupi di aggiornare l'uscita in funzione dell'ingresso. \textbf{Se la macchina è di Mealy, questo process dovrà considerare anche l'ingresso nell'assegnare l'uscita}
	\end{enumerate}
\end{itemize}

\section{Del grafico RTL}
Per disegnare un grafico RTL partendo da un codice VHDL bisogna tenere conto che
\begin{itemize}
	\item Una \textit{somma di due vector} diventa un \textit{nodo sommatore} e situazioni analoghe avverranno per le altre \textit{operazioni matematiche}
	\item I \textit{while} e gli \textit{if} diventano un \textit{MUX}
	\item Le \textit{operazioni logiche} diventano \textit{porte logiche}
	\item I \textit{signal} diventano \textit{uno o un blocco di registri}
\end{itemize}

\section{Del Bit Shift in VHDL}
\begin{itemize}
	\item Per fare il \textit{bit-shift} con numeri \textit{unsigned}, sarà sufficiente usare le funzioni \textit{shift\_left} e \textit{shift\_right}
	\item Per fare il \textit{bit-shift} con numeri \textit{signed}, bisognerà invece \textbf{preservare il MSB usando opportunamente l'operatore di concatenazione \&}. 
	\item Ad esempio, per uno \textit{shift\_right:}
	\begin{verbatim}
	    signal_shifted <= signal(signal'HIGH) & signal(signal'HIGH downto 1)
	\end{verbatim}
\end{itemize}

\section{Della metastabilità}
\begin{itemize}
	\item Corrisponde ed un livello logico non valido
	\item \`E uno stato di \textit{equilibrio instabile}
	\item Si porta a regime in uno stato di 0 o 1 valido
	\item Impiega un tempo arbitrariamente lungo per andare a regime dall'istante \(t_d\)
	\begin{itemize}
		\item La probabilità che un sistema di trovi ancora in stato metastabile tende a \(0\) esponenzialmente
		\item Per \(t \rightarrow \infty,\ P(metastabile\:a\:t)=0 \)
		\item Il modello con cui si ricava la probabilità è il \textit{Modello di Malthus} \[V-V_M \cong A \cdot e^{-\frac{t}{\tau}}\]
		con \(V\) tensione dell'anello e \(V_M\) tensione di Metastabilità
\end{itemize}
	\item Ogni sistema stabile ha almeno uno stato metastabile
	\item \textit{Il ritardo aumenta l'affidabilità.} Aggiro il problema mettendo dei registri in cascata
	\begin{enumerate}
		\item Aumentando il tempo che ci mette a propagarsi, un segnale metastabile tende a stabilizzarsi prima
		\item Aumento il tempo di \textit{clock} \(\rightarrow\) riduco la \textit{frequenza} \(\rightarrow\) riduco la possibilità di avere stati metastabili
		\item Se non posso aumentare la frequenza, aumento il numero di registri in serie
		\item \textit{Nella pratica:} 3 Flip-Flop in serie \(\rightarrow\) Probabilità nulla di metastabilità
	\end{enumerate}
\end{itemize}

\section{Della Pipeline}
\begin{itemize}
	\item In un circuito \textit{pipeline}, è possibile definire
	\begin{enumerate}
		\item \textbf{Latenza}, ritardo tra quando un segnale di ingresso è valido a quando la corrispondente uscita è valida \[Latenza = n^{\circ}\:linee\:isocrone \cdot \tau_{pd} \] con \(T_{ck}\) tempo di propagazione di un registro
		\item \textbf{Throughput}, velocità con la quale le uscite corrispondenti a nuovi ingressi diventano valide \[Throughput = \frac{1}{\tau_{pd}}\] con \(\tau_{pd}\) tempo di propagazione del blocco di pipeline più lento
	\end{enumerate}
	\item Per rendere \textit{pipeline} un circuito, bisogna
	\begin{enumerate}
		\item Creare una linea di \textit{pipeline} che interseca l'uscita
		\item Ogni linea di \textit{pipeline} deve dividere il circuito tagliando le linee di segnali in modo da essere attraversata dal segnale sempre nella stessa direzione. \textbf{Si cerca di isolare sempre la parte del circuito più lenta (quella con il maggior tempo di propagazione \( \tau_{pd} \) )}
		\item In corrispondenza dell'intersezione tra linea di livello e segnale, metto un \textit{registro}
		\item Definisco il periodo di clock minimo \(T_{ck_{min}}\) (quindi la \(f_{ck_{max}}\)) dettato dalla parte più lenta che ho isolato
	\end{enumerate}
	\item La \textit{pipeline} \textbf{peggiora} la \textbf{latenza} ma \textbf{migliora} il \textbf{throughput}
\end{itemize}
\end{document}