\documentclass{article}
\usepackage[utf8]{inputenc}
\usepackage[italian]{babel}
\usepackage{mathtools} 

% links
\usepackage[breaklinks=true]{hyperref}

\usepackage{mwe} % placeholder figure

% parskip
\usepackage[parfill]{parskip}
\setlength{\parindent}{1em}


\usepackage[most]{tcolorbox} % box colorati

\newcommand{\schr}{Schr\"odinger }
\newcommand{\ham}{\emph{Hamiltoniano }}
\renewcommand{\a}{\alpha}

\setcounter{tocdepth}{2}

\title{Elettronica dello Stato Solido\\\large Casi risolutivi dell'equazione di \schr\\
\medskip
\mbox{\url{https://valerionew.github.io/triennale-elettronica-polimi/}}
\author{Valerio Nappi}
\date{AA 2019/20}
}

\begin{document}

\maketitle

\tableofcontents
\newpage

\section{Equazione di \schr}
\label{section:intro}

L'equazione di \schr nella sua forma più generale si scrive:
\[\hat H \Psi(\vec{r},t)=i\hbar\frac{\partial \Psi}{\partial t} (\vec{r},t) \]
Dove \([\hat H]\) rappresenta l'operatore \ham applicato alla funzione d'onda \(\Psi(\vec{r},t)\), l'\ham è somma degli operatori energia cinetica \([\hat{T}]\) ed energia potenziale \([\hat{V}]\)
\[ \hat{H} = \hat{T} + \hat{V} \]
Per una particella di massa 
\[ \hat{H}  = \frac{\mathbf{\hat{p}}\cdot\mathbf{\hat{p}}}{2m}+ V(\vec{r},t) = -\frac{\hbar^2}{2m}\nabla^2+ V(\mathbf{r},t)\]
Si può quindi scrivere, nel caso più generale:
\[-\frac{\hbar^2}{2m}\nabla^2\Psi(\vec{r},t) + V(\vec{r},t)\Psi(\vec{r},t)=i\hbar\frac{\partial \Psi}{\partial t} (\vec{r},t) \]
Si noti che in generale gli operatori non commutano con la funzione a cui sono applicati (cfr. \ref{section:operatori}).

In questo corso si trattano prevalentemente casi unidimensionali. Allora, l'equazione di \schr si scrive:
\[-\frac{\hbar^2}{2m}\frac{\partial^2 }{\partial x^2}\Psi(x,t) + V(x,t)\Psi(x,t)=i\hbar\frac{\partial \Psi}{\partial t} (x,t) \]



\subsection{Valore atteso}
Definiamo il valore atteso di una \(f(x)\) come:
\[ \langle f(x) \rangle = \int\limits_{-\infty}^{+\infty}f(x) \, {|\Psi(x,t)|}^2 dx= \int\limits_{-\infty}^{+\infty}\Psi^*(x,t)\,f(x)\,\Psi(x,t) dx\]
Ad esempio definiamo il valore atteso della posizione x:
\[ \langle x \rangle = \int\limits_{-\infty}^{+\infty}x \, {|\Psi(x,t)|}^2 dx= \int\limits_{-\infty}^{+\infty}\Psi^*(x,t)\,x\,\Psi(x,t) dx\]
È possibile utilizzare il valore atteso anche per operatori applicati ad x. Ad esempio la quantità di moto \(p_x\) \textbf{per una funzione d'onda di particella libera} ha operatore \(\hat p_x = i\hbar\frac{\partial}{\partial t}\). Si potrà allora scrivere:
\[ \langle p_x \rangle = \int\limits_{-\infty}^{+\infty}\Psi^*(x,t)\,\hat p_x\,\Psi(x,t) dx = \int\limits_{-\infty}^{+\infty}\Psi^*(x,t)\,i\hbar\frac{\partial}{\partial t}\,\Psi(x,t) dx \quad \textit{(p. libera)} \]




\subsection{Operatori}
\label{section:operatori}

Come si è visto nella Sezione \ref{section:intro}, esistono alcuni operatori applicabili alle funzioni d'onda per ottenere delle grandezze fisiche. Si scriva ad esempio l'equazione di una funzione d'onda che si propaga lungo \(x\):
\[\Psi (x,t)= A e^{i(kx-\omega t)}\]
Considerando la derivata della \(\Psi (x,t)\) lungo \(x\), risulta:
\[\frac{\partial}{\partial x}\Psi (x,t)= ikA e^{i(kx-\omega t)}= i\frac{p_x}{\hbar}\Psi (x,t)\]
Si può quindi scrivere:
\[p_x = i\hbar\frac{\partial}{\partial x}\Psi (x,t)= i\hat p_x \Psi (x,t) \quad \text{con} \quad \hat p_x =i\hbar\frac{\partial}{\partial x}\]



\subsubsection{Operatori utili} 

Operatore quantità di moto:
\begin{tcolorbox}[ams equation, sharp corners]
[\hat p_x] = \left[   i\hbar\frac{\partial}{\partial x}  \right]
\end{tcolorbox}
Operatore energia totale:
\begin{tcolorbox}[ams equation, sharp corners]
[\hat E_{tot}] = \left[i\hbar\frac{\partial}{\partial t}\right]
\end{tcolorbox}
Operatore energia cinetica:
\begin{tcolorbox}[ams equation, sharp corners]
[\hat E_k] = \left[-i\frac{\hbar^2}{2m}\frac{\partial^2}{\partial t^2}\right]
\end{tcolorbox}
Operatore energia potenziale:
\begin{tcolorbox}[ams equation, sharp corners]
[\hat V] = \left[  V(x,t) \right]
\end{tcolorbox}

Ricordando poi che \(E_{tot}=E_k+V\) si può scrivere l'equazione di \schr in formato operatoriale:
\[\hat E_{tot} = \hat E_k + \hat V \quad \Longrightarrow \quad i\hbar\frac{\partial}{\partial t}=-i\frac{\hbar^2}{2m}\frac{\partial^2}{\partial t^2}+ V(x,t)\]



\subsubsection{Commutazione di operatori}
In generale non vale la proprietà commutativa tra due operatori. Si introduce allora l'operatore \emph{Commutazione di operatori} che vale, per due operatori \(\hat O\) e \(\hat P\):
\[ \left[\hat O, \hat P\right] \Psi(x,t) = \left[ \hat O \hat P-\hat P \hat O\right]  \Psi(x,t)\stackrel{?}{=}0\]

Se vale l'ultima relazione, si dice che gli operatori commutano. Si ricordi che \textbf{un operatore in generale non commuta} con la funzione \(\Psi(x,t)\). Inoltre, se vale:
\[\left[ \hat O \hat P-\hat P \hat O\right]=\left[\hat R\right]\]
Si dice che \([\hat R]\) è il commutatore degli operatori  \(\hat O\) e \(\hat P\).

\subsubsection{Autofunzioni degli operatori}
Detto \(\hat O\) un operatore, si chiama autofunzione, se esiste, la funzione \(\psi\) per cui vale:
\[ \hat O \psi = \lambda \psi\]
con \( \lambda \) uno scalare, che prende il nome di \emph{autovalore}.

Se due operatori commutano, hanno alcune autofunzioni in comune. Avere perfetta determinazine di una grandenzza individuata da un operatore, significa avere perfettamente determinata anche un'altra grandezza il cui operatore commuta col primo.

Se il commutatore tra due operatori è \textbf{non nullo}, l'indeterminazione conseguente rappresenta il \emph{Principio di indeterminazione di Heisenberg}



\subsection{Soluzione a variabili separate}
Si vuole risolvere l'equazione di \schr. Si decide di cercare soluzioni a variabili separate, cioè nella forma:
\[ \Psi(x,t) = \psi(x)\cdot \varphi (t) \]
Se \(V(x,t) = V(x)\), allora è sempre possibile trovare le \(\psi(x)\cdot \varphi (t)\). Infatti si può scrivere:
\[-\frac{\hbar^2}{2m}\frac{\partial^2\psi(x) }{\partial x^2}\varphi (t) +
V(x)\psi(x)\varphi(t)=i\hbar\frac{\partial \varphi(t)}{\partial t} \psi(x) \]
Dividendo ambo i membri per \(\psi(x)\cdot \varphi (t)\) si ottiene:
\[-\frac{\hbar^2}{2m}\frac{1}{\psi(x)}\frac{\partial^2\psi(x) }{\partial x^2} + V(x)=i\hbar\frac{1}{\varphi(t)}\frac{\partial \varphi(t)}{\partial t} = E \]
Dove il lato sinistro dipende solo dalla \(x\) e il lato destro solo dalla \(t\), e sono di conseguenza entrambi uguali ad una costante \(E\).  Si scrive quindi:
\[
\begin{cases}
E\,\psi(x) &= -\frac{\hbar^2}{2m}\frac{\partial^2\psi(x) }{\partial x^2}+V(x)\psi(x)\\
E\,\varphi(t) &=  i\hbar\frac{\partial \varphi(t)}{\partial t} 
\end{cases}
\]
Risolvendo per \(\varphi(t)\) si ottiene:
\[\varphi(t) = \varphi(0) e^{-i\frac{E}{\hbar}} \]
Che costituisce quindi solo un termine di fase.

L'equazione per \(\psi(x)\) prende invece il nome di \emph{Equazione di \schr indipendente dal tempo }, e si presenta nella forma:
\begin{equation}
\label{e:sidt}
    E\,\psi(x) = -\frac{\hbar^2}{2m}\frac{\partial^2\psi(x) }{\partial x^2}+V(x)\psi(x)
\end{equation}



\subsection{Equazione di \schr indipendente dal tempo}
Si consideri ora la (\ref{e:sidt}). Chiedersi quali funzioni risolvano l'equazione di \schr indipendente dal tempo è equivalente a chiedersi quali siano le autofunzioni dell'operatore \ham relative all'autovalore \(E\):
\[\hat H \psi(x)= E \psi(x)\]
Ci si riferirà spesso, allora, alla \(\psi(x)\) semplicemente come all'\emph{autofunzione}.




\section{Particella libera nel vuoto: onda viaggiante}
\label{section:ondaviaggiante}
Si vuole risolvere l'equazione di \schr per la funzione d'onda di una particella viaggiante liberamente nello spazio con \(V(x)\) costante.
Si disaccoppiano allora le componenti dipendente dal tempo e dipendente dallo spazio, scrivendo: \(\Psi(x,t) = \psi(x)\cdot \varphi (t)\). Si ricorda che la soluzione nel tempo risulta:
\begin{equation}
\label{e:phit}
\varphi(t) = \varphi(0)\, e^{-i\frac{E}{\hbar}t}= \varphi(0)\, e^{-i\omega t}
\end{equation}
Si cercano allora le autofunzioni di \(\hat H \psi(x)= E \psi(x)\). Il potenziale è costante, ed essendo significativo a meno di una costante, può essere assunto nullo. Si ottiene quindi :
\[-\frac{\hbar^2}{2m}\frac{\partial^2\psi(x) }{\partial x^2}= E\,\psi(x) \]
Riscritta nella forma \(\frac{\partial^2\psi(x) }{\partial x^2}=-\frac{{2mE}}{\hbar^2}\,\psi(x) =  (ik)^2\,\psi(x)\) con \( k =  \frac{\sqrt{2mE}}{\hbar} \) ha soluzione:
\begin{equation}
\label{e:phix}
\psi(x) = Ae^{ikx}+Be^{-ikx}
\end{equation}
Si tratta di due componenti viaggianti, \(Ae^{ikx}\) viaggiante verso le \(x\) positive e  \(Be^{-ikx}\) viaggiante verso le \(x\) negative. Volendo porci in una situazione in cui l'onda è generata e nota, viaggiante verso le \(x\) positive, possiamo concludere che non ci può essere una componente proveniente da \( +\infty\), e quindi \(B = 0\). Combinando la (\ref{e:phit}) e la (\ref{e:phix}), si ottiene l'\emph{onda viaggiante}:
\begin{equation}
\label{e:ondaviaggiante}
\Psi(x,t) = \psi(x)\cdot \varphi (t) = Ae^{i(kx-\omega t)}
\end{equation}




\section{Particella incidente su gradino di potenziale: onda stazionaria}
Si immagini ora di voler studiare un gradino di potenziale con espressione:
\[
    \begin{cases}
    V = 0 &\: \text{per } \: x<0 \\
    V = V_0 &\: \text{per } \; x>0 
    \end{cases}
\]
Tale che un'onda viaggiante del tipo \(Ae^{ikx}\) incida su di esso.
Si studiano ora i due casi \(E<V_0\) ed \(E>V_0\).

\begin{figure}[ht]
\centering
\includegraphics[width=\textwidth]{immagini/immagini_gradino-positivo.jpg}
\label{gradino-positivo}
\end{figure}



\subsection{Caso \(E<V_0\)}
\label{section:gradino-low-energy}
Per \(x<0\) si hanno le due onde piane del tipo \(\psi(x) = Ae^{ikx}+Be^{-ikx}\) come già visto nella Sezione \ref{section:ondaviaggiante}. 

Per \(x>0\) l'equazione di \schr tempo indipendente assume la forma:
\[\frac{\partial^2\psi(x) }{\partial x^2}=-\frac{{2m\left(E-V_0\right)}}{\hbar^2}\,\psi(x)=\a^2\,\psi(x)\]
Essendo \(E<V_0\) risulta \(\a^2=\frac{{2m\left(V_0-E\right)}}{\hbar^2}>0\). La soluzione è quindi costituita da due esponenziali \textbf{reali}, e sarà del tipo: \(\psi(x) = Ce^{\a x}+De^{-\a x}\). 
È evidente che se fosse \(C\neq0\) la soluzione divergerebbe per \(x\xrightarrow{}\infty\). Allora \(C=0\) e la soluzione per \(x>0\) risulta:
\[\psi(x) = De^{-\a x}\]
Si impongono allora le condizioni di raccordo, per stabilire il rapporto tra i coefficienti degli esponenziali:
\begin{tcolorbox}[ams equation, sharp corners]
\label{e:raccordo-gradino}
\psi\left(0^-\right)=\psi\left(0^+\right)
\end{tcolorbox}
\begin{tcolorbox}[ams equation, sharp corners]
\label{e:raccordo-gradino-dx}
\left.\frac{\partial\, \psi}{\partial x} \right|_{0^-}  =\left.\frac{\partial\, \psi}{\partial x} \right|_{0^+}
\end{tcolorbox}
Risulta quindi:
\[
\begin{cases}
 A + B = D &\xrightarrow{} \quad A = \frac{k+i\a}{2k}D \\
ikA -ikB = -\a D &\xrightarrow{} \quad B = \frac{k-i\a}{2k}D 
\end{cases}
\]
In definitiva avremo una funzione d'onda
\[ \Psi (x,t) =
\begin{cases}
 \frac{k+i\a}{2k}D e^{i\left(kx-\omega t\right)}+\frac{k-i\a}{2k}D e^{-i\left(kx-\omega t\right)} &\quad \text{per } \, x<0 \\
 De^{-\a x}e^{-i\omega t} &\quad \text{per } \, x>0 \\
\end{cases}
\]
Si osserva che nel caso classico non esiste la particella per \(x>0\). 
Inoltre,per \(x<0\), si ha espressione:
\[
\psi (x) = \frac{k+i\a}{2k}D e^{i\left(kx\right)}+\frac{k-i\a}{2k}D e^{-i\left(kx\right)} = D\left( \cos{kx}-\frac{\a}{k}\sin{kx}\right)
\]
Che è l'espressione di un'\emph{onda stazionaria}, infatti, indipendentemente dal tempo, si avranno nodi:
\[
\cos{kx}=\frac{\a}{k}\sin{kx}
\]
\subsection{Caso \(E>V_0\)}
\label{section:gradino-high-energy}
Ora per tutte le \(x\) si hanno delle onde piane. Nel dettaglio:
\[ \psi(x) = 
\begin{cases}
 Ae^{ikx}+Be^{-ikx} &\quad \text{per } \, x<0 \\
 Ce^{ik'x}+De^{-ik'x} &\quad \text{per } \, x>0 \\
\end{cases}
\]
Si noti che l'energia della particella è diversa nelle due regioni di spazio, di conseguenza esistono due vettori d'onda, \(k=\frac{\sqrt{2mE}}{\hbar}\) e \(k'=\frac{\sqrt{2m(E-V_0)}}{\hbar}\). Si noti inoltre che possiamo escludere componenti di onda viaggiante provenienti da \(+\infty\), allora \(D = 0\).
Applichiamo nuovamente le condizioni (\ref{e:raccordo-gradino}) e (\ref{e:raccordo-gradino-dx}) e otteniamo:
\[
\begin{cases}
 A + B = C &\xrightarrow{} \quad C = \frac{2k}{k+k'}A \\
ikA -ikB = -ik'C &\xrightarrow{} \quad B = \frac{k-k'}{k+k'}A 
\end{cases}
\]
Definiamo allora i coefficienti di trasmissione \(\mathbf{T}\) e di riflessione \(\mathbf{R}\)
\[
\mathbf{R}\, =\left(\frac{B}{A}\right)^2 = \left(\frac{k-k'}{k+k'}\right)^2
\]
\[
\mathbf{T} \,=\left(\frac{C}{A}\right)^2 = \left(\frac{2k}{k+k'}\right)^2
\]
Si noti infine che \(\mathbf{R}\,+\mathbf{T}\,=1\).
\newpage





\section{Particella incidente su barriera di potenziale, interferenza, risonanza}
Si vuole ora studiare la funzione d'onda di una particella incidente su una barriera di potenziale rettangolare e larga \(a\), così descritta:
\[ V =
    \begin{cases}
    \,0 &\: \text{per } \: x<0 \\
    \,V_0 &\: \text{per } \; 0<x<a \\
    \,0 &\: \text{per } \; x>a 
    \end{cases}
\]
Si tratta della somma di un gradino positivo in \(x=0\) e un gradino negativo in \(x=a\). Varranno quindi la (\ref{e:raccordo-gradino}) e la (\ref{e:raccordo-gradino-dx}). Inoltre, le stesse condizioni di raccordo varranno p in \(x=a\). Si scrive allora, ricordando la (\ref{e:raccordo-gradino}) e la (\ref{e:raccordo-gradino-dx}):
\begin{tcolorbox}[ams equation, sharp corners]
\tag{\ref{e:raccordo-gradino}}
\psi\left(0^-\right)=\psi\left(0^+\right)
\end{tcolorbox}
\begin{tcolorbox}[ams equation, sharp corners]
\tag{\ref{e:raccordo-gradino-dx}}
\left.\frac{\partial\, \psi}{\partial x} \right|_{0^-}  =\left.\frac{\partial\, \psi}{\partial x} \right|_{0^+}
\end{tcolorbox}
\begin{tcolorbox}[ams equation, sharp corners]
\label{e:raccordo-barriera}
\psi\left(a^-\right)=\psi\left(a^+\right)
\end{tcolorbox}
\begin{tcolorbox}[ams equation, sharp corners]
\label{e:raccordo-barriera-dx}
\left.\frac{\partial\, \psi}{\partial x} \right|_{a^-}  =\left.\frac{\partial\, \psi}{\partial x} \right|_{a^+}
\end{tcolorbox}



\subsection{Caso \(E<V_0\)}
Per \(E<V_0\) valgono le considerazioni già fatte nella Sezione \ref{section:gradino-low-energy}. Si cercano allora soluzioni del tipo:
\[ \psi(x) = 
\begin{cases}
 Ae^{ikx}+Be^{-ikx} &\quad \text{per } \, x<0 \\
 Ce^{-\a x}+De^{\a x} &\quad \text{per } \, 0<x<a \\
 Ee^{ikx}+Fe^{-ikx} &\quad \text{per } \, x>0 \\
\end{cases}
\]
Dove, per considerazioni analoghe a quelle delle Sezioni \ref{section:gradino-low-energy} e \ref{section:gradino-high-energy} possiamo dire che \(F=0\). Studiamo allora le condizioni di raccordo:
\[
\begin{cases}
 A + B = C + D \\
ikA - ikB = -\a C +\a D \\
Ce^{\a a}+De^{-\a a} =  Ee^{ika}  \\
- \a Ce^{\a a}+ \a De^{- \a a} = ik Ee^{ika}
\end{cases}
\]
Che risultano nel rapporto tra i coefficienti:
\[
\begin{dcases}
B = -i \dfrac{k^2+\a^2}{\a k}A\dfrac{\sinh{\a a}}{2 \cosh{\a a}+i\frac{\a^2-k^2}{\a k}\sinh{\a a}} \\
C = E \dfrac{e^{\a a}e^{ika}}{2}\left(1-\dfrac{ik}{\a}\right) \\
D = E \dfrac{e^{- \a a}e^{ika}}{2}\left(1+\dfrac{ik}{\a}\right)\\
E = 2A \dfrac{e^{-ika}}{2 \cosh{\a a}+i\frac{\a^2-k^2}{\a k}\sinh{\a a}}
\end{dcases}
\]
Si possono allora scrivere i coefficienti di trasmissione \(\mathbf{T}\) e di riflessione \(\mathbf{R}\)
\[
\mathbf{R}\, =\left|\frac{B}{A}\right|^2 = \dfrac{\dfrac{k^2+\a^2}{\a k}}{4 \cosh^2{\a a}+\left(\dfrac{\a^2-k^2}{\a k}\right)^2\sinh^2{\a a}}
\]
\[
\mathbf{T} \,=\left(\frac{C}{A}\right)^2 = \dfrac{4}{4 \cosh^2{\a a}+\left(\dfrac{\a^2-k^2}{\a k}\right)^2\sinh^2{\a a}} = \dfrac{1}{1+\left(\dfrac{\a^2+k^2}{\a k}\right)^2\sinh^2{\a a}} 
\]
Per \(\a\, a \gg 1\) è poi possibile approssimare 
\[
\mathbf{T} \approx \left(\dfrac{4\a k}{\a^2+k^2}\right)^2 \: e^{-2\a a} \approx e^{-2\a a} 
\]

\subsection{Caso \(E>V_0\)}
\section{Particella confinata in buca di potenziale infinita, autovalori}

\section{Particella confinata in buca di potenziale finita, soluzione grafica}
\section{Particella confinata in buca di potenziale infinita 2D}
\section{Doppia buca accoppiata}

\end{document}
